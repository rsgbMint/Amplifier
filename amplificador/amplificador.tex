\documentclass[a4paper, 12pt]{article}
\usepackage[obeyspaces]{url}

%\usepackage{etoolbox}
\include{source_code/preamble.tex}

\begin{document}
\begin{titlepage}
     \newcommand{\HRule}{\rule{\linewidth}{0.5mm}}
     \center
     \textsc{\LARGE  Universidad Nacional de San \\[0.2cm] Antonio Abad del Cusco}\\[1.5cm]
	 \includegraphics[scale=0.3]{figures/escudo.jpg}\\[1cm]
     \textsc{\Large Facultad de Ingeniería Eléctrica, \\Electrónica, Informática y Mecánica}\\[0.5cm]
     \textsc{\large Escuela Profesional de Ingeniería Electrónica}\\[5mm]
	 
	 \HRule \\[0.4cm]
     { \huge \bfseries CIRCUITOS ELECTRÓNICOS I}\\[1.8mm] 
	 \HRule\\[5mm]
     \begin{minipage}{\textwidth}
        \center
        \textit{Trabajo final}\\[0.2cm]
        \text{\large Diseño de un amplificador}\\[3mm]
		\begin{tabular}{ll}
        \emph{Docente:} \\[1.8mm]
            \hspace{1.4cm}\textsc{Ing. Milton Jhon Velasquez Curo}\\[3mm]
        \emph{Estudiantes:} & \quad \emph{Códigos:}\\[1.8mm]
            \hspace{1.4cm} \textsc{Roly Sandro Gutierrez Benito} & \hspace{1.7cm} 182967\\[3mm]
            \hspace{1.4cm} \textsc{Edison Abado Ancco} & \hspace{1.7cm} 145012\\[3mm]
            \hspace{1.4cm} \textsc{Alberto Arturo Quiñones Pacco}& \hspace{1.7cm} 160901\\[3mm]
		 \end{tabular} \\[3.5mm]
		 \textsc{CUSCO-PERÚ}\\
         \textsc{2022}
     \end{minipage}
\end{titlepage}
\newpage
\tableofcontents

\newpage
\section{Diseñar e implementar un amplificador de señales}
Las características del amplificador de señales son los siguientes:
\begin{itemize}
    \item Señal de entrada pico $V_{ip} = 0.1 - 1\ \text{mV}$
    \item Ganancia de voltaje $A_v = 80\ \text{dB}$
    \item Impedancia de salida $Z_o = 8\ \Omega$
    \item Rango de frecuencias $f_l = 40 \text{Hz}$, $f_h = 40\ \text{kHz}$ 
\end{itemize}

\section{Etapas del amplificador}

\subsection{Preamplificador}
Para esta etapa, se hizo uso de un amplificador multietapa.
\begin{itemize}
    \item \textbf{Primera etapa}
        \begin{itemize}
            \item Polarización universal en emisor común.
            \item Transistor \verb|BC548A|
        \end{itemize}

    \item \textbf{Segunda etapa}
        \begin{itemize}
            \item Polarización universal en emisor común.
            \item Transistor \verb|BC548A|
        \end{itemize}
\end{itemize}

\newpage
\subsection{Diseño y cálculos para la primera etapa}
Dado el siguiente circuito en \textbf{polarización universal} en \textbf{emisor común}:  
\begin{figure}[H]
    \centering
    \includegraphics[scale=0.5]{figures/image1.png}
    \caption{Primera etapa del preamplificador con polarización universal EC}
    \label{fig:image1}
\end{figure}

Para el transistor \verb|BC548A|, se obtuvieron los siguientes valores según \textbf{datasheet}:
\begin{itemize}
    \item $\text{I}_{\text{C}_{\text{máx}}} = 100\ \text{mA}$
    \item $\text{h}_{\text{fe}} = 150$
    \item $\text{V}_{\text{CE}} = 5\ \text{V}$
    \item $\text{P}_\text{D} = 625\ \text{mW}$
    \item $V_{BE} = 0.7\ \text{V}$
\end{itemize}
Ahora, se mostrarán las ecuaciones que serán usadas para el diseño 
\begin{align}
    V_{CEQ} &= \dfrac{V_{CC}}{2} \label{eq:1}\\[1mm]
    V_{CEQ} &= I_{CQ}\cdot R_{AC} \label{eq:2}\\[1mm]
    R_B &\le 0.1\cdot \beta \cdot R_E \label{eq:3}\\[1.5mm]
    P_D &= V_{CEQ}\cdot I_{CEQ} \label{eq:4}\\[1.5mm]
    R_1 &= \dfrac{V_CC - V_{BE} -I_E\cdot R_E}{10I_B} \label{eq:5}\\[1.5mm]
    R_2 &= \dfrac{V_{BE} + (\beta + 1)I_B\cdot R_E}{10I_B}\label{eq:6}\\[1.5mm]
    I_C &= \beta I_B \label{eq:7}
\end{align}  

proponiendo $V_{CC} = 18 V$, de \eqref{eq:1}:
\begin{align*}
    V_{CEQ} &= \dfrac{18}{2} \\[1.5mm]
    \therefore V_{CEQ} &= 9 V
\end{align*}

ahora, de \eqref{eq:4} y del \textbf{\textit{datasheet}}
\begin{align*}
    P_D &= V_{CEQ}\cdot I_{CQ} \\[1mm]
    I_{CQ} &= \dfrac{600m}{9V} \\[1mm]
    \therefore I_{CQ} &= 33.33 mA 
\end{align*}

en \eqref{eq:2}
\begin{align*}
    V_{CEQ} = I_{CQ} \cdot R_{AC} \\[1mm]
    R_{AC} = \dfrac{V_{CEQ}}{I_{CQ}} \\ = \dfrac{9V}{33.33mA}\\[1.5mm]
    \therefore R_{AC} = 0.270K
\end{align*}
se tiene que $R_{AC} = R_E + R_C$, se propone $R_E = 10 \Omega$
$$ \Rightarrow R_C = 0.260 k\Omega.$$

Para determinar los valores de $R_1$ y $R_2$, se tiene en \eqref{eq:7}: 
\begin{align*}
    I_B &= \dfrac{I_C}{h_{ie}}\\[1.5mm]
    I_B &= \dfrac{33.33 mA}{150}\\[1.5mm] 
    \therefore I_B &= 222.22 \mu A
\end{align*}

de \eqref{eq:5} y \eqref{eq:6}:
\begin{align*}
    R_1 &= \dfrac{V_{CC} - V_{BE} -I_E\cdot R_E}{10\cdot I_B}\\[1.5mm]
    R_1 &= \dfrac{18 - 0.7 - (33.33m)(0.01k)}{10\cdot 222.22\mu}\\[1.5mm]
    \therefore R_1 &= 7.63 k\Omega \\[2mm]
    R_2 &= \dfrac{V_{BE} + (\beta + 1)I_B\cdot R_E}{10\cdot I_B}\\[1.5mm]
    R_2 &= \dfrac{0.7 + (151)(222.22\mu)(0.01k)}{10\cdot 222.22\mu}\\[1.5mm]
    \therefore R_2 &= 0.15 k\Omega
\end{align*}

\subsubsection{Gráfica del punto de operación}
De la malla de salida se tiene:
\begin{align}
    V_{CC} = R_C\cdot I_C + R_E\cdot I_C
\end{align}

con las siguientes condiciones:
\begin{align*}
    \Bigl. V_{CE}\Bigr|_{I_C=0} &= V_{CC}\\[1.5mm]
    \Rightarrow V_{CE} &= 18 V \\[3mm]
    \Bigl. I_{C}\Bigr|_{V_{CE}=0} &= \dfrac{V_{CC}}{R_C + R_E}\\[1.5mm]
    \Rightarrow I_{C} &= 66.67 mA
\end{align*}
con los datos obtenidos, el valor de punto Q es: 
$I_{CQ} = 33.33 mA$ y $ V_{CEQ} = 9 V$

\begin{pycode}
import matplotlib.pyplot as plt

V_CE = 18 
I_C = 66.67 
V_CEQ = 9
I_CQ = 33.33

label = f"Q({V_CEQ} V, {I_CQ} mA)"
DC = [[0,V_CE],[I_C,0]]

fig = plt.figure()
plt.plot(DC[0],DC[1])
plt.plot(V_CEQ, I_CQ, "o")
plt.annotate(label, (V_CEQ, I_CQ), 
textcoords="offset points", xytext=(0,-10), ha="center")

plt.grid(True)
plt.xlabel("$V_{CE}\ (V)$")
plt.ylabel("$I_{C}\ (mA)$")
fig.savefig("figures/plot1.pdf")
\end{pycode}
\begin{figure}[H]
    \centering
    \includegraphics[scale=0.5]{figures/plot1.pdf}
    \caption{Punto de operación primera etapa del préamplificador}
\end{figure}

\newpage
\subsubsection{Rectas de carga en DC y AC}
Haciendo el análisis en DC y AC, se tienen los circuitos equivalentes:

\begin{figure}[H]
    \centering
    \includegraphics[scale=0.4]{figures/image2.png}
    \caption{Análisis en DC en la malla de salida}
\end{figure}

de esa malla, se tiene:
\begin{align}
    V_{CC} &= R_C\cdot I_C + R_E\cdot I_E \\
    I_E &\approx I_C \nonumber\\
    V_{CC} &= I_C(R_C + R_E) \nonumber\\
    \therefore R_{DC} &= R_C + R_E \nonumber
\end{align}

para la carga en AC, se tiene
\begin{figure}[H]
    \centering
    \includegraphics[scale=0.4]{figures/image3.png}
    \caption{Análisis en AC en la malla de salida}
\end{figure}
en la malla de salida
\begin{align*}
    \Delta v_{CE} &= -\Delta i_C (R_E + R_C) \\
    \therefore R_{AC} &= R_E + R_C
\end{align*}

entonces, se tiene la ecuación para la recta de carga en AC
\begin{equation}
    i_C = -\dfrac{v_{CE}}{R_{AC}} + \dfrac{V_{CEQ}}{R_{AC}} + I_{CQ}
\end{equation}

con las condiciones siguientes
\begin{align*}
    \Bigl. i_{\text{cmáx}}\Bigr|_{v_{CE}=0} &= \dfrac{V_{CEQ}}{R_{AC}} + I_{CQ} \\[1.5mm]
    \Rightarrow i_{\text{cmác}} &= 67.95 mA \\[3mm]
    \Bigl. v_{\text{CEmáx}}\Bigr|_{i_C=0} &= V_{CEQ} + I_{CQ}\cdot R_{AC}\\[1.5mm]
    \Rightarrow v_{\text{CEmáx}} &= 17.67 V
\end{align*}

\begin{pycode}
import matplotlib.pyplot as plt

# DC
V_CE = 18 
I_C = 66.67 

# AC
v_CE = 17.67 
i_C = 67.95

# Punto Quiescente Q
V_CEQ = 9
I_CQ = 33.33

label = f"Q({V_CEQ} V, {I_CQ} mA)"
DC = [[0,V_CE],[I_C,0]]
AC = [[0, v_CE], [i_C,0]]

fig = plt.figure()

plt.plot(DC[0],DC[1],label="recta DC")
plt.legend()
plt.plot(V_CEQ, I_CQ, "o")
plt.plot(AC[0],AC[1],label="recta AC")
plt.legend()

plt.annotate(label, (V_CEQ, I_CQ), 
textcoords="offset points", xytext=(0,-10), ha="center")

plt.grid(True)
plt.xlabel("$V_{CE}\ (V)$")
plt.ylabel("$I_{C}\ (mA)$")
fig.savefig("figures/plot2.pdf")
\end{pycode}

\begin{figure}[H]
    \centering
    \includegraphics[scale=0.5]{figures/plot2.pdf}
    \caption{Rectas de carga en AC y DC}
\end{figure}

\subsubsection{Gráfica de Bode}

\subsubsection{Potencia}
La potencia máxima que se consume por parte del transistor es:
\begin{align*}
    \text{P}_\text{D} &= \text{V}_{\text{CEQ}}\cdot \text{I}_{\text{CQ}} \\
    \text{P}_\text{D} &= (9\text{V})(33.33 \text{mA}) \\
    \therefore \text{P}_\text{D} &= 300 \text{mW}.
\end{align*}

La potencia suministrada será:
\begin{align*}
    P_{CC} = V_{CC}\cdot I\\
    P_{CC} = V_{CC}\cdot (I_{CQ} + I_1)\\
    P_{CC} = V_{CC}\cdot \left[I_{CQ} + \dfrac{V_{CC}}{R_1 + R_2}\right]\\[1.5mm]
    \therefore P_{CC} = 620.76 \text{mW}
\end{align*}

La potencia que se entrega a la carga $\text{R}_\text{L}$, en este caso será igual a la 
impedancia de entrada de la segunda etapa ($\text{Z}_{\text{i2}}$).
\begin{figure}[H]
    \centering
    \includegraphics[scale=0.5]{figures/image4.png}
    \caption{Modelo híbrido simplificado}
    \label{fig:image5}
\end{figure}

Determinando el $h_{ie}\ $:
\begin{align*}
    h_{ie} &= \dfrac{V_T}{I_B}\\[1mm]
    \Rightarrow h_{ie} &= 112.5 \Omega\\
\end{align*}

de la malla de salida de la Figura \eqref{fig:image5}, se tiene que (sin $\text{R}_\text{L}$)
\begin{align*}
    A_v &= \dfrac{-h_{fe}(R_C)}{h_{ie}+(hfe+1)} \\[1.5mm]
    A_v &= \dfrac{-150(260)}{112.5+150+1}\\[1mm]
    \therefore \text{A}_\text{v} &= -148
\end{align*}

\begin{align*}
    \text{P}_\text{L} &= \dfrac{\text{V}_{\text{o}}^2}{\text{R}_\text{L}}\\[1.5mm]
    \text{P}_\text{L} &= \dfrac{\left(\dfrac{\text{V}_\text{o}}{\sqrt{2}}\right)}
    {\text{R}_\text{L}}
\end{align*}

\begin{align*}
    \eta = \dfrac{P_L}{P_{CC}}
\end{align*}
\subsection{Diseño y cálculos para la segunda etapa}
\subsubsection{Gráficas del punto de operación}
\subsubsection{Rectas de carga en DC y AC}
\subsubsection{Gráfica de Bode}
\subsubsection{Potencia}

\newpage
\subsection{Amplificador de potencia}
En esta etapa, se tienen las siguientes características:
\begin{itemize}
    \item Amplificador clase B
\end{itemize}

\subsubsection{Diseño y cálculos}
\subsubsection{Gráficas del punto de operación}
\subsubsection{Rectas de carga en DC y AC}
\subsubsection{Gráfica de Bode}
\subsubsection{Potencia}
\end{document}

